
\documentclass[submit]{harvardml}

% Put in your full name and email address.
\name{Your Name}
\email{email@fas.harvard.edu}

% List any people you worked with.
\collaborators{%
  John Doe,
  Fred Doe
}

% You don't need to change these.
\course{CS181-S16}
\assignment{Assignment \#3}
\duedate{5:00pm March 25, 2016}

\usepackage[OT1]{fontenc}
\usepackage[colorlinks,citecolor=blue,urlcolor=blue]{hyperref}
\usepackage[pdftex]{graphicx}
\usepackage{subfig}
\usepackage{fullpage}
\usepackage{palatino}
\usepackage{mathpazo}
\usepackage{amsmath}
\usepackage{amssymb}
\usepackage{color}
\usepackage{todonotes}
\usepackage{listings}
\usepackage{common}
\usepackage{bm}

\usepackage[mmddyyyy,hhmmss]{datetime}

\definecolor{verbgray}{gray}{0.9}

\lstnewenvironment{csv}{%
  \lstset{backgroundcolor=\color{verbgray},
  frame=single,
  framerule=0pt,
  basicstyle=\ttfamily,
  columns=fullflexible}}{}

\begin{document}
\begin{center}
{\Large Homework 3: SVM}\\
\end{center}

There is a mathematical component and a programming component to this homework.
Please submit ONLY your PDF to Canvas, and push all of your work to your Github
repository. If a question requires you to make any plots, like Problem 3, please
include those in the writeup.

%%%%%%%%%%%%%%%%%%%%%%%%%%%%%%%%%%%%%%%%%%%%%
% Problem 1
%%%%%%%%%%%%%%%%%%%%%%%%%%%%%%%%%%%%%%%%%%%%%
\begin{problem}[Fitting an SVM by hand, 8pts]
Consider a dataset with the following 6 points in $1D$: \[\{(x_1, y_1)\} =\{(-3
, +1 ), (-2 , +1 ) , (-1,  -1 ), ( 1 , -1 ), ( 2 , +1 ), ( 3 , +1 )\}\] Consider
mapping these points to $2$ dimensions using the feature vector $\phi : x
\mapsto (x, x^2)$. The max-margin classifier objective is given by:
\begin{equation}
  \min_{\mathbf{w}, w_0} \|w\|_2^2 \quad \text{s.t.} \quad
  y_i(\mathbf{w}^T\phi(x_i) + w_0) \geq 1,~\forall i
\end{equation}

Note: the purpose of this exercise is to solve the SVM without the help of a
computer, relying instead on principled rules and properties of these
classifiers. The exercise has been broken down into a series of questions, each
providing a part of the solution. Make sure to follow the logical structure of
the exercise when composing your answer and to justify each step.

\begin{enumerate}
  \item Write down a vector that is parallel to the optimal vector $\mathbf{w}$.
    Justify your answer.
  \item What is the value of the margin achieved by $\mathbf{w}$? Justify your
    answer.
  \item Solve for $\mathbf{w}$ using your answers to the two previous questions.
  \item Solve for $w_0$. Justify your answer.
  \item Write down the discriminant as an explicit function of $x$.
\end{enumerate}

\end{problem}
\subsection*{Solution}




\newpage
%%%%%%%%%%%%%%%%%%%%%%%%%%%%%%%%%%%%%%%%%%%%%
% Problem 2
%%%%%%%%%%%%%%%%%%%%%%%%%%%%%%%%%%%%%%%%%%%%%
\begin{problem}[Composing Kernel Functions, 7pts]
Prove that
\begin{align*}
	K(\boldx, \boldx') &= \exp\{ -||\boldx - \boldx'||^2_2 \}\,,
\end{align*}
where~$\boldx,\boldx'\in\reals^D$ is a valid kernel, using only the following
properties.  If~$K_1(\cdot,\cdot)$ and~$K_2(\cdot,\cdot)$ are valid kernels,
then the following are also valid kernels:
\begin{align*}
	K(\boldx, \boldx') &= c\,K_1(\boldx, \boldx') \quad \text{for $c>0$}\\
	K(\boldx, \boldx') &= K_1(\boldx, \boldx') + K_2(\boldx, \boldx')\\
	K(\boldx, \boldx') &= K_1(\boldx, \boldx')\,K_2(\boldx, \boldx')\\
	K(\boldx, \boldx') &= \exp\{ K_1(\boldx, \boldx') \}\\
  K(\boldx, \boldx') &= f(\boldx)\,K_1(\boldx, \boldx')\,f(\boldx') \quad
  \text{where $f$ is any function from~$\reals^D$ to $\reals$}
\end{align*}

 \end{problem}
\subsection*{Solution}


\newpage
%%%%%%%%%%%%%%%%%%%%%%%%%%%%%%%%%%%%%%%%%%%%%
% Problem 3
%%%%%%%%%%%%%%%%%%%%%%%%%%%%%%%%%%%%%%%%%%%%%
\begin{problem}[Scaling up your SVM solver, 10pts]
We will release Problem 3 shortly!
\end{problem}

\subsection*{Solution}



\newpage

\subsection*{Calibration [1pt]}
Approximately how long did this homework take you to complete?


\end{document}


















































